\documentclass[11pt]{article}
\usepackage{amsmath,amssymb,amsthm}

\DeclareMathOperator*{\E}{\mathbb{E}}
\DeclareMathOperator*{\var}{Var}
\let\Pr\relax
\DeclareMathOperator*{\Pr}{\mathbb{P}}

\newcommand{\eps}{\epsilon}
\newcommand{\abs}[1]{|#1|}
\newcommand{\norm}[1]{\|#1\|}
\newcommand{\inprod}[1]{\left\langle #1 \right\rangle}
\newcommand{\R}{\mathbb{R}}

\newcommand{\handout}[5]{
  \noindent
  \begin{center}
  \framebox{
    \vbox{
      \hbox to 5.78in { {\bf CS : Deep Learning } \hfill #2 }
      \vspace{4mm}
      \hbox to 5.78in { {\Large \hfill #5  \hfill} }
      \vspace{2mm}
      \hbox to 5.78in { {\em #3 \hfill #4} }
    }
  }
  \end{center}
  \vspace*{4mm}
}

\newcommand{\lecture}[4]{\handout{#1}{#2}{#3}{Scribe: #4}{Lecture #1}}

\usepackage{framed}
\newtheorem{theorem}{Theorem}
\newtheorem{corollary}[theorem]{Corollary}
\newtheorem{lemma}[theorem]{Lemma}
\newtheorem{observation}[theorem]{Observation}
\newtheorem{proposition}[theorem]{Proposition}
\newtheorem{definition}[theorem]{Definition}
\newenvironment{example}[1][\unskip]
{\begin{framed}\textbf{Example: #1.}}
{\end{framed}}
\newtheorem{claim}[theorem]{Claim}
\newtheorem{fact}[theorem]{Fact}
\newtheorem{assumption}[theorem]{Assumption}

% 1-inch margins, from fullpage.sty by H.Partl, Version 2, Dec. 15, 1988.
\topmargin 0pt
\advance \topmargin by -\headheight
\advance \topmargin by -\headsep
\textheight 8.9in
\oddsidemargin 0pt
\evensidemargin \oddsidemargin
\marginparwidth 0.5in
\textwidth 6.5in

\parindent 0in
\parskip 1.5ex

\begin{document}

\lecture{1 --- September 2, 2020}{Fall 2020}{Prof.\ Zhao Song}{Zhao Song}

\section{Private Deep Learning}

\subsection{Two hidden layer neural network}

Let $\phi : \R \rightarrow \R$ denote the ReLU activation, i.e., $\phi(u) = \max\{u,0\}$.

We consider a two hidden layer neural network $f : \R^d \rightarrow \R$ (which can be decomposed into two functions $h : \R^d \rightarrow \R^{m_a}$ and $g : \R^{m_a} \rightarrow \R^{m_b}$) as follows:
\begin{align*}
    h(x) = & ~ \phi( W_A x ) , \\
    g(z) = & ~ W_C^\top \phi (z) , \\
    f(x) = & ~ g( h( x ) ) = W_C^\top \phi( W_B \phi ( W_A x) ) ,
\end{align*}
where $W_A \in \R^{m_a \times d}$, $W_B \in \R^{m_b \times m_a}$ and $W_C \in \R^{m_b}$.

Let $\{ (x_1, y_1), (x_2, y_2), \cdots, (x_n,y_n) \} \subset \R^d \times \R$ denote a set of $n$ input data points. We can think of $x_i$s are images and $y_i$s are the corresponding labels.

In classical deep learning training, the task is to find weights $W_A, W_B$ and $W_C$ such that ${\cal L}$ is minimized.
\begin{align*}
{\cal L} = \sum_{i=1}^n \| f(x_i) - y_i \|_2^2 
\end{align*}

In this lecture, we describe a slightly different goal.
The purpose is to find some $W_A$, $W_B$ and $W_C$ such that
\begin{enumerate}
  \item Utility: $f(x_i) \approx y_i$, $\forall i \in [n]$
  \item Privacy: Given $h(x_i)$ and $W_A$, it is ``hard'' recover $x_i$
\end{enumerate}

\end{document}